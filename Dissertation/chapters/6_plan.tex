\chapter{Project Plan}
% todo words

\section{Analysis}
The project plan was created in Trello as a series of lists representing each week. These weeks then contained several tasks that were to be completed, following the Scrum methodology laid out in section \ref{sec:methodology}. The original development plan can be seen in Appendix \ref{appendix:plan}, which lists the sprints for each week of semester 2 up until when the report was to be started. As for semester 1, the tasks consisted purely of research and the construction of concept applications.

A lot of thought went into the planning stage, with each sprint's tasks trying to be as detailed and extensive as possible. This was difficult to perform without knowing the exact scope of each component in the system, but it turned out to be an excellent reference for managing time. With each completed component in the system, the backlog (or lack of) indicated roughly how far behind or ahead the project was.

\section{Outcome}
While the original project plan was a good reference, the development of the project didn't follow it entirely. The development of the storage server started about a week late, and ended up taking much longer than expected. What should have been completed by week 3 was instead completed around the end of week 4. The development of the storage server would have taken much longer if it weren't for the research completed into the areas of Git and GitHub.

The delay from the storage server pushed all of the other tasks laid out back by about 10 days, meaning the user testing happened in week 9 rather than week 7. There was also a change in the way the project was to be deployed to AWS, which was going be done throughout development. Instead, the deployment was left until development was complete which continued with the AWS service structure in mind. The research into the AWS services for selecting technologies aided in keeping the project compatible with Elastic Beanstalk, which only accepts certain versions of Python.

Due to time constraints, several features were dropped from the project, including the Web Audio API integration, and the visual merging tool on the web application. If this project was repeated, more time would have been allocated towards development during semester 1.
