\documentclass[a4paper, 12pt]{article}
\usepackage[a4paper]{geometry}
\usepackage{microtype}
\usepackage{graphicx}
\usepackage{parskip}
\usepackage{hyperref}

% Set the images directory
\graphicspath{{images/}}

% Show links in the TOC
\hypersetup{hidelinks}

% Disable line breaks
\setlength\righthyphenmin{62}
\setlength\lefthyphenmin{62}

% Disable paragraph breaks
\widowpenalties 1 10000

% Start writing the document
\begin{document}

% Title page
\title{
	{Cloud Game Engine}\\
	\large{Interim Report}\\
	% \large{Dublin Institute of Technology}\\
	% Student Number: C11426252\\
	% Supervisor: Brian Keegan
	% TODO
	% http://alvinalexander.com/blog/post/latex/reference-figure-or-table-within-latex-document
}
\author{Declan Curran}
\maketitle
\newpage

% Table of Contents
\setcounter{tocdepth}{3}
\tableofcontents
\newpage

% Document contents
\section{Project Statement}
\section{Research}
\subsection{Background Research}

Web-based applications are becoming increasingly popular as the web develops; they are often favoured over traditional desktop applications when it comes to things like writing documents (Google Drive, Office Live) or backing up files. Desktop applications will often require downloading large installation files, dependencies (Java, .NET), and can often be limited to a single platform. Web apps are now becoming the norm, as they are far more flexible, can be updated and distributed instantly, and with the rise of HTML5 and cloud storage; web apps are becoming extremely powerful and pleasing to use.

% Ref: Web becoming popular: Johan Harjono, Gloria Ng, Ding Kong, and Jimmy Lo. 2010. Building smarter web applications with HTML5.


While web apps are becoming extremely popular, there are still many types of applications that haven't been created for the web. Game engine suites such as Unity, CryEngine or Unreal Engine still require downloading a huge application to be able create games for the desktop. This can make getting into game development seem very complicated for new users. A web application could be created in this area to make it easier for new game developers, easing game development by creating, publishing and playing games all in the cloud.

% Ref: Unity, CryEngine, Unreal


Building a 3D game engine in the browser requires significant performance over normal JavaScript usage, which wouldn't be possible without HTML5 and WebGL. 

% Ref: http://www.reedbushey.com/85Webgl%20Up%20and%20Running.pdf


WebGL is an open-source API for rendering 3D graphics in the browser, which is based on OpenGL; a commonly used graphics processing API that is used in a lot of games today. WebGL is quickly becoming ``the new standard for 3D graphics on the Web'' (Parisi, 2012) as it is the most powerful and flexible way to render graphics in-browser without extra dependencies or plugins. JavaScript is used to tell WebGL what to do, but it can be rather complicated to program; ``To do anything more than the most basic tasks using [WebGL] out of the box requires serious effort and literally hundreds of lines of code.'' (Parisi, 2014)

% Fix Parisi reference.
% Ref: OpenGL


Because WebGL is so complicated to start out with, it often turns programmers with no OpenGL experience off of making games using it. The solution to WebGL's complexity is to use a library that wraps WebGL and simplifies the interface. The Three.js JavaScript library accomplishes this by providing easy to use classes such as models, lighting, scenes and cameras. ``A program written in raw WebGL style using hundreds of lines of code can be expressed in just a few dozen lines of code with Three.js'' (Parisi, 2014)

% Ref parisi book first sentence
% Ref quote
% Ref: Three.js


While Three.js greatly simplifies the process of rendering scenes in the browser, it's not good enough for the development of games, as it only provides a framework for rendering without any game processes. Game engines such as Unity are able to easily build scenes and entities by simply dragging and dropping of components onto entities to create scenes with minimal programming required. 

% Ref: Unity


The architecture of the game engine is extremely important for how accessible it is to developers. The Entity-Component-System is a pattern used to simplify how game objects are defined by separating behaviour into smaller components. Each game object can be made up of many components, for example; a cube object could have a model component, giving it something to render, and a physics component which will allow it to collide with the world and other physics objects. This type of game engine architecture allows users who have minimal programming skill to modify the game logic easily.

% Ref: ECS
% http://t-machine.org/index.php/2007/09/03/entity-systems-are-the-future-of-mmog-development-part-1/

% Talk about editor?


By combining the aspects of an Entity Component System into a game engine built on the web with a fully featured editor and cloud storage, inexperienced or hobbyist game developers will be able to easily create games and publish them without needing to invest a lot of time in installing software, learning APIs and just get straight to making their game.

\subsection{Alternative Existing Solutions}
\subsubsection{Unity}
Unity is a 3D game engine with an integrated editor and a huge online library full of assets such as models, sounds and various scripts or tools. Unity is a cross-platform engine that is used on Windows, Linux, Mac OSX, mobiles (Android, iOS, Windows, BlackBerry), consoles (PS3, Xbox360, Wii U) and even in browsers.

% Integrated editor: http://unity3d.com/unity/workflow/integrated-editor
% Multi-platform: http://unity3d.com/unity/multiplatform

Unity's ability to publish games to the web via the Unity Web Player browser plugin make it a powerful solution to creating games for the web. The only downside to the web player is that the user must install the runtime plugin to be able to play.

% Web Player: http://unity3d.com/unity/multiplatform/web

The Unity game engine is based on the Entity-Component-System pattern, with the majority of components used being supplied by the engine (Such as lighting, physics, models). Unity games can be scripted in one of three languages; C\#, Boo or JavaScript.

% Ref scripting: http://docs.unity3d.com/ScriptReference/

The Unity scene editor allows complex entities to be created simply by adding a number of components to an empty entity in a drag-and-drop style. The properties for each component can then be edited in the \emph{properties inspector}, such as changing the model or colour of an object. 

% Add pic of inspector
% Ref editor again?

\subsubsection{PlayCanvas}
PlayCanvas is a browser based game engine and editor with cloud storage. 

% Ref: https://playcanvas.com

\subsubsection{CopperLicht}
``CopperLicht is an open source WebGL library and JavaScript 3D engine for creating games and 3D applications in the webbrowser. It uses the WebGL canvas supported by modern browsers and is able to render hardware accelerated 3d graphics without any plugins.''

% Quote Ref: http://www.ambiera.com/copperlicht/

The CopperLicht game engine uses an OOP style approach to the architecture of a game, with entities being defined via scripting rather than by selecting a group of components as Unity does. The CopperLicht engine will handle rendering, physics and animations, any extra functionality must be added manually by the user with a script.

% Ref: CopperLicht

The CopperLicht world editor, \emph{CopperCube} is an editor designed to create the worlds used in a CopperLicht game. CopperCube is only available for Windows and Mac.

% Ref: http://www.ambiera.com/copperlicht/features.html

When a user is finished creating their world, CopperCube creates a world file and generates a basic HTML page to render the scene. The game logic script is added to this HTML page manually by the user, where they can add user interactions via HTML events (keydown, mousedown).

% Ref: http://www.ambiera.com/copperlicht/documentation/tutorials/tutorial-02.html

\subsection{Technologies Researched}
\subsubsection{Web Application Hosting}
\paragraph{Criteria.}
TODO. This project requires a powerful and scalable hardware solution, with virtualization, automatic scaling, load balancing ...

% TODO ref

\paragraph{Dedicated Server.}
The first option is to use a dedicated server without any distributed computing. For this case I would use my own server which is provided by Hetzner; Root Server EX40. As I would be using my own server, this option would be the most cost-effective, however the performance and scalability of the platform would not be scalable enough for the final release of the project.

% Ref: Hetzner EX40

It would be possible to make this option viable by manually setting up load balancing and other performance increasing tools, but it would be too much work.

% Ref: memcached paper, performance etc
% Ref: http://www.hetzner.de/de/hosting/produkte_rootserver/ex40

\paragraph{Amazon Elastic Compute Cloud.}
Amazon Elastic Compute Cloud (EC2), part of the Amazon Web Services (AWS) collection, is one of the largest \emph{Infrastructure as a Service} cloud services in production today. It is used by hundreds of large companies to host their web applications or even game servers. EC2 provides scalable virtual servers which can TODO

% Ref AWS, EC2

\paragraph{Amazon EC2 Container Service.}
Amazon EC2 Container Service (ECS) brings EC2 one step further by making deployment as simple as possible. Docker.io \emph{dockerfiles} are used to automatically create EC2 instances based on a specification created in a simple text file. These instances can then be populated with apps and data specified in the dockerfile.

% Ref: Docker, ECS

Using ECS would significantly speed up the time required to set up the necessary EC2 instances, especially if multiple instances are required; such as a reverse proxy, application and a database.

\paragraph{Heroku.}
Heroku is a \emph{Platform as a Service} web application platform for deploying scalable web apps in the cloud. Heroku will automatically scale and manage applications deployed on the platform, making sure the application performs perfectly even under stress. Heroku has support for deploying applications built in many languages including Java, Node.js and Python.

% Ref Heroku

\paragraph{Amazon Elastic Beanstalk.}
Amazon Elastic Beanstalk is a web application deployment service similar to Heroku. Elastic Beanstalk takes a web app and deploys it to an EC2 instance, automatically setting up and managing load balancing, scaling and any capacity needs.

% Ref: http://aws.amazon.com/elasticbeanstalk/

Using Elastic Beanstalk in my project would make deployment quick and painless, but may not be fully suitable as it does not have the same level of control as deploying EC2 instances manually. This would be more suited to a simple web application without extra layers such as a database.


\subsubsection{Web Application Frameworks}
\paragraph{Criteria.}
\begin{itemize}
\item{TODO: Write shit here instead of bullets}
\item{Should be able to handle JSON data from the client well.}
\item{Lightweight}
\item{Support for HTTP requests}
\item{Doesn’t need built-in database functionality (bloat)}
\item{MVC}
\item{Templating}
\item{Flexible}
\item{Performance}
\end{itemize}

\paragraph{Node.js.}
Node.js is a platform for building applications using JavaScript as the main language. It uses Google's V8 JavaScript engine to increase the performance of JavaScript by compiling standard JavaScript to machine code.

% Ref Node

By default, Node.js isn't a web application framework, but can be made into a powerful one using one of many libraries available on the Node Package Manager. An example of this would be using the Express library and a template engine such as EJS or Jade.

% Ref NPM
% Ref http://expressjs.com/
% Ref http://paularmstrong.github.io/node-templates/


\paragraph{Django.}
Django is a web application framework built on Python, which uses an MVC-like approach to designing websites. Django is designed to allow common web development tasks to be completed quickly and easily. It does this by providing APIs for Object-Relational Modelling to ease database usage, a powerful templating engine and automatic administration interface generation.

% Ref Django
% Ref common: https://docs.djangoproject.com/en/1.7/intro/overview/
% Ref ORM: https://docs.djangoproject.com/en/1.7/intro/overview/
% Ref admin: https://docs.djangoproject.com/en/1.7/intro/tutorial02/

While Django makes web application development simple, it is bloated with features that may not be used at all in a lot of projects.?? todo

\paragraph{Flask.}
Flask is lightweight Python-based web application microframework that comes only with a server and a templating engine. Flask is designed to be relatively simple at it's core, but becomes powerful through the use of modules that add extra functionality if needed. The simplistic nature of Flask means it can be up and running relatively quick compared to other frameworks.

% Ref Flask: http://flask.pocoo.org/


\paragraph{JavaServer Pages.}
JavaServer Pages (JSP) is ... todo

% Ref JSP

Can’t handle JSON very well

bloated



\subsubsection{Project Storage}
\paragraph{Criteria.}
User project storage is an extremely important part of this project. How the projects are stored will define how the application performs while saving projects, serving project files to players, and managing how users will collaborate on projects.

The project storage medium should be able to support some kind of version control, have good read/write performance, and should be scalable to some extent.

\paragraph{GitHub API.}
Storing user projects via the GitHub API without any server-side storage was the original proposed approach for this project. All storage would be done directly through API requests to GitHub by submitting the user's files without any storage on disk. This provides both version control and the cloud-based user storage this project requires.

% Ref github
% Ref github API

There are a few problems with this approach. First is that it is very volatile; changes made by the user may be discarded if the user leaves the page without committing. Secondly, using the GitHub API does not do any merging of files, but instead overwrites files entirely, making collaboration hard for the users. This is discussed in greater detail in the Prototyping and Development area of this report (Section 6).

% Ref Json?

\paragraph{Amazon Elastic Block Storage.}
Amazon Elastic Block Storage (EBS) is a scalable solution to storage for use on Amazon EC2 instances. Performance on EBS is very high, as there are options for selecting between high/normal performance SSDs and standard hard drives. EBS volumes can be mounted directly to EC2 instances allowing standard file storage without any APIs.

% Ref AWS EBS

While EBS has no form of version control, local Git repositories could be used to manage version control. Since EBS can be mounted directly onto an EC2 instance, the command-line Git tool can be used to directly perform Git operations such as committing or even pushing to a remote server such as GitHub.

% Ref Git

\paragraph{Amazon Simple Storage Service.}
Amazon Simple Storage Service (S3) is a storage system that is accessible through a REST-like API, which can be used with the AWS SDK which is available for several languages. S3 stores data as objects inside a container called a \emph{bucket}. S3 does not have any version control mechanisms, so version control would need to be implemented manually.

\paragraph{MongoDB GridFS.}
MongoDB GridFS is a high performance file system based on the MongoDB document database. It is designed to be used when documents that are larger than 16MB are to be stored in a MongoDB database, but can be used for small documents for consistency without any significant performance loss. 

% Ref GridFS: http://docs.mongodb.org/manual/core/gridfs/

Because GridFS is based on MongoDB, it is extremely suitable for horizontal scaling, but it does not support any atomic operations. The lack of atomic operations could make collaboration and version control hard to implement.

% Ref atomic: http://docs.mongodb.org/manual/faq/developers/


\subsubsection{Alternative Scripting Languages}
\paragraph{Criteria.}
Alternative browser-based scripting languages could be used to add typing and Object Oriented Programming to the game engine and user made games. While JavaScript has limited OOP functionality (Prototype chaining for inheritance), it is still a valid choice.

\paragraph{Dart.}
Dart is a compiled JavaScript alternative created by Google, which supports OOP principals and static typing. Dart can also be compiled to JavaScript which allows dynamic loading of either the Dart file or the JavaScript file depending on what is supported by the browser. Dart supports interaction with standard JavaScript libraries such as jQuery without needing to be modified in any way.

Dart runs in it's own Dart VM which boasts huge performance gains over JavaScript virtual machines, however the Dart VM is currently only supported on the \emph{Dartium} browser; a branch of Google Chrome.

% Ref Dart

\paragraph{TypeScript.}
TypeScript is another alternative to JavaScript that was created by Microsoft to replace JavaScript. TypeScript is compiled directly to JavaScript without the need for a different virtual machine. In order to use existing JavaScript libraries with TypeScript, a header file is required to make the functions visible to TypeScript.

% Ref TS

\subsubsection{JavaScript Libraries}

\paragraph{Rendering.}
Three.js

SpiderGL

SceneJS

\paragraph{Physics.}
Comparison of Physics Frameworks for WebGL-Based Game Engine 

Resa Yogya and Raymond Kosalaa

% http://www.epj-conferences.org/articles/epjconf/pdf/2014/05/epjconf_icas2013_00035.pdf




\subsection{Other Relevant Research}

\subsection{Resultant Findings / Requirements}
\subsection{Bibliography}
\section{Analysis}
\section{Approach and Methodology}
% Github, Trello
\section{Design}
\subsection{Technical Architecture}
\section{Prototyping and Development}
\section{Testing}
\section{Issues and Risks}
% Using Git with OAuth tokens
\section{Plan and Future Work}
\section{Conclusions}
\end{document}
